\subsection{Метод простых итераций}

Построим новую функцию \(g(x)\), что будет сжимающим отображением, для него нужно будет решить уже следующее уравнение:
\begin{align}
  g(x) = x 
\end{align}

Для того, чтобы это отображение было сжимающим, нужно удовлетворить условию сжатия. То есть, нужно построить следующую Липшецеву функцию:
\begin{align}
  \forall x, y \in X \colon |g(x) - g(y)| \leq K |x - y| < |x - y|
\end{align}

Построим функцию \(g\) следующим образом:
\begin{align}\label{eq:g_def}
  g(x) = x + c f(x)
\end{align}
Работаем с функцией \(f \in C^1 \mathbb{R} \) --- можем свести все к \(g \in C^1 \mathbb{R} \), воспользоваться теоремой Лагранжа. Если \(\|g'\| \leq K < 1\), то какие две точки мы бы не взяли, мы получаем сжатие:
\begin{align}
  \exists \xi \in X \colon |g(x) - g(y)| = |g'(\xi)| |x - y| \leq K |x - y| < |x - y|
\end{align}

Значит при том условии, что \(\|g'\| \leq K < 1\) мы в точности получаем сжимающее отображение \(g\). Осталось подобрать нужное для того \( c \).
\begin{align}
  \forall x \in X \colon |g'(x)| = |1 + c f'(x)| < 1
\end{align}
Раскрывая модуль, получаем оценку для числа \(c\):
\begin{align}
  \forall x \in X \colon -\frac{2}{f'(x)} < c < 0
\end{align}

А из того, что \(\forall x \in X \colon f(x) > 0\) следует то, что нужно взять значение \(c\) такое, что:
\begin{align}
  -\frac{2}{\sup\limits_{x\in X} f'(x)} < c < 0
\end{align}

Из того что функция у нас выпукла наверх на отрезке, мы возьмем в качестве такого \(x\) левую границу окрестности, \(-1\). \(f'(-1)\) можно оценить сверху как:
\begin{align}
  f'(-1) = \frac{2}{e^2} + 3 < \frac{1}{2} + 3 < 1 + 3 = 4
\end{align}
Проведя такую оценку, получаем промежуток для \(c\):
\begin{align}
  \left[-\frac{1}{2}, 0\right) \subset \left(-\frac{2}{\sup\limits_{x\in X} f'(x)} , 0\right)
\end{align}

Нам можно взять любое \(c\) из данного промежутка, так что возьмем:
\begin{align}
  c = -\frac{1}{2} 
\end{align}

Итак, мы получаем сжимающее отображение:
\begin{align}
  \varphi(x) = x - \frac{1}{2} f(x) = x - \frac{1}{2} (e^{2x} + x^3)
\end{align}

Отсюда и итерационная формула:
\begin{align}\label{eq:sim_iter}
  x_{k + 1}  = x_k - \frac{1}{2}(e^{2x} + x_k^3)
\end{align}

Будем считать, что мы можем остановить алгоритм при совместном выполнении условий:
\begin{align}\label{eq:sim_stop}
  \begin{cases}
    |x_k - x_{k - 1}| \leq \varepsilon \\
    |f(x_k)| \leq \delta 
  \end{cases} 
\end{align}

Предлагаю посмотреть на то, как ведут себя два соседних приближения, члена последовательности. Определить меру их сближения в сравнении с предыдущими членами.
\begin{align}
  |\xi - x_{k + 1}| = |g(\xi) - g(x_k)| \leq \sup_{x_\in X}|1 + c f'(x)| |\xi - x_k|
\end{align}
Причем это выполняется, по построению, сразу с номера \(k = 0\). Мы сходимся <<линейно>>, уменьшая расстояние между корнем и приближением в \(K\) раз. Отсюда видно, что выбор \(c\) непосредственно влияет на то, через сколько шагов мы достигнем достаточной оценки корня.

\subsubsection{Резюме}
\begin{enumerate}
  \item Итерационные формула, описанная в \cref{eq:sim_iter}
  \begin{align*}
    x_{k + 1}  = x_k - \frac{1}{2}(e^{2x} + x_k^3)
  \end{align*}
\item Условие остановки алгоритма, опписанная в \cref{eq:sim_stop}
\begin{align*}
  \begin{cases}
    |x_k - x_{k - 1}| \leq \varepsilon \\
    |f(x_k)| \leq \delta 
  \end{cases} 
\end{align*}

\end{enumerate}
