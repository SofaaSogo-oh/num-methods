\subsection{Метод Зейделя}
Данный метод является модификацией метода итерации. Основан он на том, что при вычислении \( (k+1) \)-го приближения неизвестного \(x_i\) учитываются вычисленные прежде \(x_i^{(k + 1)}\) для всех \(j = \overline{1,\, i}\).

Приводя конкретные шаги итерации, можно привести метод к следующему виду:

\begin{align}\label{eq:seidel-iter}
  x_i^{(k + 1)} = b_i + \sum_{j=1}^{i - 1}a_{ij}x_j^{(k + 1)} + \sum_{j=i}^{n}a_{ij}x_j^{(k)}
\end{align}

Попробуем, основываясь на методе итерации, разобраться с методом Зейделя, а именно с его сходимостью. Сразу обозначим нужную задачу --- нам нужно показать, что выполняется оценка, что выполняется сжатие:
\begin{align*}
  \rho(A x, A x^{(k)}) \leq \alpha \rho(x, x^{(k-1)})
\end{align*}

Рассмотрим расстояние между приблиением \(x^{(k)}\) и \(x\). Сперва разобьем требуемое на части --- вычислим \(|x_i - x_i^{(k)}|\):
\begin{align}
  |x_i - x_i^{(k)}| = \sum_{j=1}^{i - 1} a_{ij}(x_j - x_j^{(k)}) + \sum_{j=i}^n a_{ij}(x_j - x_j^{(k-1)}) \leq \sum_{j=1}^{i - 1} |a_{ij}||x_j - x_j^{(k)}| + \sum_{j=i}^n |a_{ij}||x_j - x_j^{(k-1)}|
\end{align}

И, работая с метрикой \(\rho(x, y) = \max |x_i - y_i|\), мы можем оценить расстояние между координатами через расстояние между точками:
\begin{align}
  |x_i - x_i^{(k)}| \leq \rho(x, x^{(k)})
\end{align}

И тогда мы можем заключить, что
\begin{equation}
  |x_i - x_i^{(k)}| \leq \sum_{j=1}^{i - 1} |a_{ij}| \rho(x, x^{(k)}) + \sum_{j=i}^n |a_{ij}| \rho(x, x^{(k - 1)}) 
\end{equation}

Мы отделили сумму коэффициентов от поокординатой разности, заменив на общее расстояние --- мы можем упростить выражение, взяв \(\sum\limits_{j=1}^{i - 1}|a_{ij}|=a_{i}\) и \(\sum\limits_{j=i}^{n}|a_{ij}|=b_{i}\). Раз оценка  действительна для любого \(i\), мы можем взять значение по максимуму --- получаем оценку для метрики. Ну и возьмем соотвественные \(a_i = a\), \(b_i = b\).

\begin{align}
  \rho(x, x^{(k)}) \leq a \rho(x, x^{(k)}) + b \rho(x, x^{(k-1)}) \Leftrightarrow \rho(x, x^{(k)}) \leq \frac{b}{1 - a} \rho(x, x^{(k - 1)})
\end{align}

Мы почти получили требуемую оценку --- нам нужно показать, что \(\frac{b}{1 - a} < 1 \). В этом нам помогает то, что \(a + b = \sum\limits_{j=1}^n |a_{ij}|\). Нам понадобится уже использованная при методе простых итераций оценка (13). При такой оценке ясно, что \(a + b < 1\) (это следует из того, что мы берем максимум среди всех строк в оценке (13)). Тогда \(b < 1 - a\). И мы получаем:
\begin{align}
  \frac{b}{1 - a} < \frac{1 - a}{1 - a} = 1
\end{align}

И мы теперь можем найти такое \(\alpha < 1\), что 
\begin{align*}
  \rho(x, x^{(k)}) \leq \alpha \rho(x, x^{(k - 1)})
\end{align*}

А это уже условие сжатия --- мы показали, что метод Зейделя сходится к исходному решению. Для него действительны все оценки, проводимые при методе простых итераций.

\subsubsection*{Резюме}
Итак, подытожим рассуждения. Исходя из сказанного, мы имеем:
\begin{enumerate}
  \item Итерационные формулы. Перепишем \eqref{eq:seidel-iter}:
\begin{align}
  \begin{cases}
    x^{(k+1)} = -\frac{1}{3} + \frac{1}{3} y^{(k)} + \frac{1}{3} z^{(k)} \\
    y^{(k+1)} = 1 + \frac{1}{2} x^{(k+1)} + \frac{1}{4} z^{(k)} \\
    z^{(k+1)} = -\frac{1}{2} + \frac{1}{3} x^{(k+1)} + \frac{1}{2} y^{(k+1)} 
  \end{cases} 
\end{align}
  \item Условие окончания. Такое же, как и в \ref{sec:sim}
  \begin{align}
    |x^{(k + 1)} - x^{(k)}| + |y^{(k + 1)} - y^{(k)}| + |z^{(k + 1)} - z^{(k)}| \leq \varepsilon
  \end{align}
\end{enumerate}
