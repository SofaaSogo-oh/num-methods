\section{Результаты работы программы}

\begin{table}[h!]
	\begin{tabular}{|p{12ex}|l|p{20ex}|p{20ex}|p{20ex}|}
		\hline
		Начальное приближение                     & Точность                            & Метод простых итераций & Метод Ньютона & Модифицированный метод Ньютона \\ \hline
		\(x^{(0)} = -1 \)                         &                                                                                                               %
		\(\varepsilon = \delta = 10^{-3} \)       &                                                                                                               %
		\(x^{(k+1)}=-0.648833\) \newline \(n=5\)  &                                                                                                               %
		\(x^{(k+1)}=-0.648844\) \newline \(n=4\)  &                                                                                                               %
		\(x^{(k)}=-0.6491089\) \newline \(n=8\)                                                                                                                   \\ \cline{2-5}
		                                          & \(\varepsilon = \delta = 10^{-5} \) &                                                                         %
		\(x^{(k+1)}=-0.648844\) \newline \(n=7\)  &                                                                                                               %
		\(x^{(k+1)}=-0.648844\) \newline \(n=5\)  &                                                                                                               %
		\(x^{(k)}=-0.648849\) \newline \(n=13\)                                                                                                                   \\ \hline

		\(x^{(0)} = -0.7\)                        &                                                                                                               %
		\(\varepsilon = \delta = 10^{-3} \)       &                                                                                                               %
		\(x^{(k+1)}=-0.64887 \) \newline \(n=3 \) &                                                                                                               %
		\(x^{(k+1)}=-0.648844\) \newline \(n=3 \) &                                                                                                               %
		\(x^{(k)}=-0.648856\) \newline \(n=3 \)                                                                                                                   \\ \cline{2-5}
		                                          & \(\varepsilon = \delta = 10^{-5} \) &                                                                         %
		\(x^{(k+1)}=-0.648844\) \newline \(n=5\)  &                                                                                                               %
		\(x^{(k+1)}=-0.648844\) \newline \(n=3 \) &                                                                                                               %
		\(x^{(k)}=-0.648844\) \newline \(n=5 \)                                                                                                                   \\ \hline

		\(x^{(0)} = -0.6\)                        &                                                                                                               %
		\(\varepsilon = \delta = 10^{-3} \)       &                                                                                                               %
		\(x^{(k+1)}=-0.648784\) \newline \(n=3 \) &                                                                                                               %
		\(x^{(k+1)}=-0.648844\) \newline \(n=3 \) &                                                                                                               %
		\(x^{(k)}=-0.648855\) \newline \(n=3 \)                                                                                                                   \\ \cline{2-5}
		                                          & \(\varepsilon = \delta = 10^{-5} \) &                                                                         %
		\(x^{(k+1)}=-0.648844\) \newline \(n=5\)  &                                                                                                               %
		\(x^{(k+1)}=-0.648844\) \newline \(n=3 \) &                                                                                                               %
		\(x^{(k)}=-0.648844\) \newline \(n=5 \)                                                                                                                   \\ \hline
	\end{tabular}
\end{table}

\section{Выводы}
\begin{enumerate}
	%\item Выбранного метода (метод Зейделя для данного примера показал себя лучше всего);
	%\item От выбора начального приближения (при более сильном отклонении нужно проводить больше итераций в силу того, что идет оценка в степень сжатия с фиксировнным начальным расстоянием: \(\rho(x_{n}, x) \leq \alpha^n \rho(x_0, x_1)\));
	%\item От выбора заданной точности (при том замечание: в заданном примере \(\alpha > \frac{1}{2}\), поэтому необходимо учитывать и её в рассчетах, это было продемонстрировано на примере метода простых итераций --- здесь оценка аргумента и условие остановки вычислений отличаются).
	\item Метод Ньютона имеет наибольшую скорость сходимости;
	\item Чем больше точность, тем меньше скорость сходимости;
	\item Чем ближе находится начальное приближение к точному решению, тем больше скорость сходимости;
\end{enumerate}
