\section{Результаты работы программы}

\begin{table}[h!]
	\begin{tabular}{|p{12ex}|l|p{20ex}|p{20ex}|p{20ex}|}
		\hline
		Начальное приближение                                      & Точность                            & Метод простых итераций & Метод Ньютона & Модифицированный метод Ньютона \\ \hline
		\(x^{(0)} = 1.4 \) \newline \(y^{(0)} = 4.2\)              &                                                                                                               %
		\(\varepsilon = \delta = 10^{-3} \)                        &                                                                                                               %
		\(x^{(k+1)}=1.5\) \newline \(y^{(k+1)}=4\) \newline\(n=3\) &                                                                                                               %
		\(x^{(k+1)}=1.5\) \newline \(y^{(k+1)}=4\) \newline\(n=3\) &                                                                                                               %
		\(x^{(k+1)}=1.49991\) \newline \(y^{(k+1)}=4.00017\) \newline\(n=5\)                                                                                                       \\ \cline{2-5}
		                                                           & \(\varepsilon = \delta = 10^{-5} \) &                                                                         %
		\(x^{(k+1)}=1.5\) \newline \(y^{(k+1)}=4\) \newline\(n=4\) &                                                                                                               %
		\(x^{(k+1)}=1.5\) \newline \(y^{(k+1)}=4\) \newline\(n=4\) &                                                                                                               %
		\(x^{(k+1)}=1.5\) \newline \(y^{(k+1)}=4\) \newline\(n=9\)                                                                                                                 \\ \hline

		\(x^{(0)} = 1.1 \) \newline \(y^{(0)} = 4.9\)              &                                                                                                               %
		\(\varepsilon = \delta = 10^{-3} \)                        &                                                                                                               %
		\(x^{(k+1)}=1.5\) \newline \(y^{(k+1)}=4\) \newline\(n=5\) &                                                                                                               %
		\(x^{(k+1)}=1.5\) \newline \(y^{(k+1)}=4\) \newline\(n=5\) &                                                                                                               %
		\(x^{(k+1)}=1.49944\) \newline \(y^{(k+1)}=4.00112\) \newline\(n=12\)                                                                                                      \\ \cline{2-5}
		                                                           & \(\varepsilon = \delta = 10^{-5} \) &                                                                         %
		\(x^{(k+1)}=1.5\) \newline \(y^{(k+1)}=4\) \newline\(n=5\) &                                                                                                               %
		\(x^{(k+1)}=1.5\) \newline \(y^{(k+1)}=4\) \newline\(n=5\) &                                                                                                               %
		\(x^{(k+1)}=1.49999\) \newline \(y^{(k+1)}=4.00001\) \newline\(n=22\)                                                                                                      \\ \hline
		\(x^{(0)} = 0.8\) \newline \(y^{(0)} = 5.1\)               &                                                                                                               %
		\(\varepsilon = \delta = 10^{-3} \)                        &                                                                                                               %
		\(x^{(k+1)}=1.5\) \newline \(y^{(k+1)}=4\) \newline\(n=5\) &                                                                                                               %
		\(x^{(k+1)}=1.5\) \newline \(y^{(k+1)}=4\) \newline\(n=5\) &                                                                                                               %
		\(x^{(k+1)}=1.49916\) \newline \(y^{(k+1)}=4.00169\) \newline\(n=16\)                                                                                                      \\ \cline{2-5}
		                                                           & \(\varepsilon = \delta = 10^{-5} \) &                                                                         %
		\(x^{(k+1)}=1.5\) \newline \(y^{(k+1)}=4\) \newline\(n=6\) &                                                                                                               %
		\(x^{(k+1)}=1.5\) \newline \(y^{(k+1)}=4\) \newline\(n=6\) &                                                                                                               %
		\(x^{(k+1)}=1.49999\) \newline \(y^{(k+1)}=4.00002\) \newline\(n=30\)                                                                                                      \\ \hline
	\end{tabular}
\end{table}

% 1.49916 4.00169 

\section{Выводы}
\begin{enumerate}
	%\item Выбранного метода (метод Зейделя для данного примера показал себя лучше всего);
	%\item От выбора начального приближения (при более сильном отклонении нужно проводить больше итераций в силу того, что идет оценка в степень сжатия с фиксировнным начальным расстоянием: \(\rho(x_{n}, x) \leq \alpha^n \rho(x_0, x_1)\));
	%\item От выбора заданной точности (при том замечание: в заданном примере \(\alpha > \frac{1}{2}\), поэтому необходимо учитывать и её в рассчетах, это было продемонстрировано на примере метода простых итераций --- здесь оценка аргумента и условие остановки вычислений отличаются).
	\item Метод Ньютона и метод простых итераций имеют наибольшую скорость сходимости.
	\item Чем больше точность, тем меньше скорость сходимости.
	\item Чем ближе находятся начальные приближения к точному решению системы нелинейных уравнений, тем больше скорость сходимости рассматриваемых итерационных процессов.
\end{enumerate}
