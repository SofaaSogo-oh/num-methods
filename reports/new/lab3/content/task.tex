\section{Цель работы}
научиться находить приближённое решение краевой задачи для дифференциальных уравнений в частных производных гиперболического типа методом сеток на ЭВМ.

\section{Содержание работы}
\begin{enumerate}
	\item Изучить метод сеток для дифференциальных уравнений в частных производных гиперболического типа;
	\item Заменить исходное уравнение, используя конечно-разностную аппроксимацию;
	\item Составить программу для получения приближённого численного решения краевой задачи для уравнения свободных колебаний струны.
	\item Решить задачу с помощью ЭВМ.
	\item Составить отчёт о работе.
\end{enumerate}

\section{Выполнение работы}
\subsection{Задание}
Получить приближённое численное решение краевой задачи, используя функции
\begin{align}
	f(x) = \gamma e^{m x} + \cos{\gamma x},   \\
	F(x) = \gamma e^x + \beta \cos{\gamma x}, \\
	\varphi(t) = \alpha t + \sin{\beta t},    \\
	\psi(t) = e^{N t} + M \sin{m t} + N
\end{align}

Провести вычисления значений функции \(u(x,t)\) с помощью составленной программы, если \(l = 1\), \(n = 10\), \(j = \overline{0, 10}\).

\subsection{Решение}
Пусть дано уравнение свободных колебаний струны
\begin{align} \label{eq:init-u-diff}
	\frac{\partial^2 u}{\partial t^2} = a^2 \frac{\partial^2 u}{\partial x^2}
\end{align}
где \(u(x,t)\) ---  положение точки струны с координатой \(x\) в момент времени \(t\).
С начальными условиями:
\begin{align}\label{eq:init-cond}
	u(x, 0) = f(x), \\
	\label{eq:init-cond-prt}
	\frac{\partial u(x,0)}{\partial t} = F(x)
\end{align}

И граничными условиями:
\begin{align}\label{eq:bnd-cond}
	\begin{cases}
		u(0,t) = \varphi(t), \\
		u(l,t) = \psi(t)
	\end{cases}
\end{align}

Требуется найти положение струны \(u(x,t)\) в любой точке \(x \in [0,l]\)  в любой момент времени \(t\). Для решения этой задачи используется метод сеток.

В методе сеток в полуполосе \(t \geq 0, 0 \leq x \leq l\) строится сетка, узлами которой являются точки \((x_i, t_j)\), где \(x_i = i \cdot h\), \(i = \overline{0,n}\), \(t_j=j\cdot k\), \(j = 0,1,2\dots\), \(h = \frac{l}{n}\), \(k = \frac{h}{a}\).

Приняв \(u_{i,j} = u(x_i,t_j)\) и заменив \cref{eq:init-u-diff}, конечно-разностными соотношениями, получается
\begin{align}
	\frac{u_{i,j+1} - 2 u_{i,j} + u_{i,j-1}}{k^2} = a^2 \frac{u_{i+1,j} - 2 u_{i,j} + u_{i - 1,j}}{h^2}, \forall i = \overline{0,n}, j = 0,1,2,\dots
\end{align}
А подставив \(k = \frac{h}{a}\)
\begin{align}\label{eq:simple-u-diff}
	u_{i,j+1} = u_{i+1,j} + u_{i - 1,j} - u_{i,j - 1}
\end{align}

Таким образом, для подсчёта значения искомой функции \(u(x_i, t_j)\) в узловых точках слоя \((j + 1)\) используются уже известные значения этой функции в двух узловых точках слоя \(j\), а также  в одной узловой точке слоя \((j - 1)\). Для вычислений по формуле \cref{eq:simple-u-diff} необходимо знать значения \(u_{i,0}\) и \(u_{i,-1}\).

Начальное условие \cref{eq:init-cond} задаёт значения \(u_{i,j}\) в слое \(j = 0\). Определим значения в слое \(j = -1\), используя начальное условие \cref{eq:init-cond-prt}. Запишем это условие в конечноразностной форме
\begin{align}
	\frac{u_{i,0} - u_{i,-1}}{k} = F_i
\end{align}
Отсюда --- <<виртуальный>> \(j=-1\) слой

\begin{align}
	u_{i,-1} = u_{i,0} - k F_i, i = \overline{1,n-1}
\end{align}

Исходя из \(u_{i,j}\) при \(j = 0\) и \(j = -1\), а также используя \cref{eq:bnd-cond} производятся вычисления \(u_{i,j}\) по \cref{eq:simple-u-diff}.

\subsubsection*{Для рассматриваемого примера}

\begin{align}
	f(x) = -1.7 e^{-1.7 x} + \cos(-1.7 x), \\
	F(x) = -0.5 e^x + 0.4 \cos(-1.7 x),    \\
	\varphi(t) = -0.5 t + \sin{0.4 t},     \\
	\psi(t) = e^{-0.5 t} + 0.1 \sin(-1.7 t) - 0.5
\end{align}

Начальные  условия
\begin{align}
	u(x, 0) = -1.7 e^{-1.7 x} + \cos(-1.7 x), \\
	u_t'(x, 0)= -0.5 e^x + 0.4 \cos(-1.7 x)
\end{align}

Граничныме условия
\begin{align}
	\begin{cases}
		u(0,t) = -0.5 t + \sin{0.4 t}, \\
		u(l,t) = e^{-0.5 t} + 0.1 \sin(-1.7 t) - 0.5
	\end{cases}
\end{align}

Начальное условие в конечно-разностных соотношениях
\begin{align}
	u_{i,0} = -1.7 e^{-1.7 x_i} + \cos(-1.7 x_i) \\
	u_{i,-1} = u_{i,0} - k F_i = -1.7 e^{-1.7 x_i} + \cos(-1.7 x_i) - k (-0.5 e^x + 0.4 \cos(-1.7 x)
	)
\end{align}

Граничные условия в конечно-разностных соотношениях:
\begin{align}
	u_{0,j} = -0.5 t_j + \sin{0.4 t_j}, \\
	u_{n,j} = e^{-0.5 t_j} + 0.1 \sin(-1.7 t_j) - 0.5
\end{align}

\subsection{Листинг программы}
\inputminted[frame=lines, linenos]{cpp}{listings/main.cc}

\subsection{Результаты работы программы}

\section{Вывод}
Составлена программа, позволяющая вычислять значение функции \(u(x,t)\), которая определяет положение точки \(x_i\) струны длины \(l\) в любой момент времени \(t_j\).

