\section{Цель работы}
Научиться решать обыкновенные дифференциальные уравнения методами Эйлера, Рунге-Кутта и Адамса с помощью ЭВМ.

\section{Содержание работы}
\begin{enumerate}
	\item Изучить методы Эйлера, Рунге-Кутта и Адамса для приближенного решения задачи Коши.
	\item На конкретном примере усвоить порядок решения обыкновенного дифференциального уравнения указанными методами с помощью ЭВМ.
	\item Составить программу на любом языке программирования, реализующую процесс приближенного решения обыкновенного дифференциального уравнения указанными методами.
	\item Сделать вывод о точности используемых методов.
	\item Составить отчет о проделанной работе.
\end{enumerate}

\section{Задание}
\begin{enumerate}
	\item Аналитически решить задачу Коши вида:
	      \begin{align}
		      \label{eq:diff-eq}
		      \frac{dy}{dx} = f(x, y) = y \cos x \\
		      \label{eq:beg-cnd}
		      y(x_0) = y(0) = y_0 = 1
	      \end{align}
	\item Записать рабочие формулы метода Эйлера, метода Рунге-Кутта четвертого порядка точности и метода Адамса для численного решения уравнения \cref{eq:diff-eq} с начальным условиями \cref{eq:beg-cnd} на отрезке
	      \begin{align}\label{eq:x-src}
		      x \in [x_0, x_n] = [0, 1]
	      \end{align}
	\item Составить программу на любом языке программирования, реализующую построенные процессы
\end{enumerate}
\section{Решение}
\subsection{Аналитическое решение задачи Коши}
Параметризировав \(x, y\) как
\begin{align}
	\begin{cases}
		x(t) = t \\
		y(t) = y(x)
	\end{cases}
\end{align}
И рассматривая уравнение
\begin{align}
	r(x, y) = p(x) - q(y) = 0
\end{align}
При нахождении производной
\begin{align}
	r'_t = r'_x x'_t + r'_y y'_t = 0 \\
	r'_x + r'_y y' = 0 \Longrightarrow y' = -\frac{r'_x}{r'_y} = \frac{p'_x}{q'_y} = y \cos x
\end{align}
Тогда пусть
\begin{align}
	p'_x = \cos x \Longrightarrow p(x) = \sin x + c \\
	q'_y = y^{-1} \Longrightarrow q(y) = \ln y + c
\end{align}
Тогда
\begin{align}
	\ln y = \sin x + c \Longrightarrow y = c_1 e^{\sin x}
\end{align}
Смотря на \cref{eq:beg-cnd}
\begin{align}
	y(0) = 1 \Longrightarrow c_1 e^{\sin 0} = c_1 e^0 = c_1 \cdot 1 = c_1 = 1
\end{align}
Итак, решение задачи Коши \cref{eq:diff-eq} с начальными условиями \cref{eq:beg-cnd}:
\begin{align}
	y = e^{\sin x}
\end{align}
\subsection{Приближение решения}
Для построения рабочих формул методов Эйлера, Рунге-Кутта четвертого порядка точности и Адамса разделим отрезок \cref{eq:x-src} на \(n\) равных частей и сформируем систему равноотстоящих точек
\begin{align}
	x_{i+1} = x_i + h, i = \overline{0, n - 1}
\end{align}
При том \(x_0 = 0\), \(x_n = 1\), шаг
\begin{align}
	h = \frac{x_n - x_0}{n} = \frac{1}{n}
\end{align}
\subsubsection{Метод Эйлера}
