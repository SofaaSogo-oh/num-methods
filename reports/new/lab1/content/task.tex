\section{Цель работы}
Научиться решать обыкновенные дифференциальные уравнения методами Эйлера, Рунге-Кутта и Адамса с помощью ЭВМ.

\section{Содержание работы}
\begin{enumerate}
	\item Изучить методы Эйлера, Рунге-Кутта и Адамса для приближенного решения задачи Коши.
	\item На конкретном примере усвоить порядок решения обыкновенного дифференциального уравнения указанными методами с помощью ЭВМ.
	\item Составить программу на любом языке программирования, реализующую процесс приближенного решения обыкновенного дифференциального уравнения указанными методами.
	\item Сделать вывод о точности используемых методов.
	\item Составить отчет о проделанной работе.
\end{enumerate}

\section{Задание}
\begin{enumerate}
	\item Аналитически решить задачу Коши вида:
	      \begin{align}
		      \label{eq:diff-eq}
		      \frac{dy}{dx} = f(x, y) = y \cos x \\
		      \label{eq:beg-cnd}
		      y(x_0) = y(0) = y_0 = 1
	      \end{align}
	\item Записать рабочие формулы метода Эйлера, метода Рунге-Кутта четвертого порядка точности и метода Адамса для численного решения уравнения \cref{eq:diff-eq} с начальным условиями \cref{eq:beg-cnd} на отрезке
	      \begin{align}\label{eq:x-src}
		      x \in [x_0, x_n] = [0, 1]
	      \end{align}
	\item Составить программу на любом языке программирования, реализующую построенные процессы
\end{enumerate}
\section{Решение}
\subsection{Аналитическое решение задачи Коши}
Параметризировав \(x, y\) как
\begin{align}
	\begin{cases}
		x(t) = t \\
		y(t) = y(x)
	\end{cases}
\end{align}
И рассматривая уравнение
\begin{align}
	r(x, y) = p(x) - q(y) = 0
\end{align}
При нахождении производной
\begin{align}
	r'_t = r'_x x'_t + r'_y y'_t = 0 \\
	r'_x + r'_y y' = 0 \Longrightarrow y' = -\frac{r'_x}{r'_y} = \frac{p'_x}{q'_y} = y \cos x
\end{align}
Тогда пусть
\begin{align}
	p'_x = \cos x \Longrightarrow p(x) = \sin x + c \\
	q'_y = y^{-1} \Longrightarrow q(y) = \ln y + c
\end{align}
Тогда
\begin{align}
	\ln y = \sin x + c \Longrightarrow y = c_1 e^{\sin x}
\end{align}
Смотря на \cref{eq:beg-cnd}
\begin{align}
	y(0) = 1 \Longrightarrow c_1 e^{\sin 0} = c_1 e^0 = c_1 \cdot 1 = c_1 = 1
\end{align}
Итак, решение задачи Коши \cref{eq:diff-eq} с начальными условиями \cref{eq:beg-cnd}:
\begin{align}
	y = e^{\sin x}
\end{align}
\subsection{Приближение решения}
Для построения рабочих формул методов Эйлера, Рунге-Кутта четвертого порядка точности и Адамса разделим отрезок \cref{eq:x-src} на \(n\) равных частей и сформируем систему равноотстоящих точек
\begin{align}
	x_{i+1} = x_i + h,\quad i = \overline{0, n - 1}
\end{align}
При том \(x_0 = 0\), \(x_n = 1\), шаг
\begin{align}
	h = \frac{x_n - x_0}{n} = \frac{1}{n}
\end{align}
\subsubsection{Метод Эйлера}
Приближенная интегральная кривая заменяется ломаной, состоянщей из \(A_n(x_i, y_i), i = \overline{0, n-1}\). Наклон каждого звена \(A_iA_{i+1}\), а, соответственно, положение точки \(A_{i+1}\) определятся из приближенной замены конечной разностью
\begin{align}
	\frac{y_{i+1} - y_i}{h} = \Delta y(x_i) \approx y'(x_i) = f(x_i, y_i) \\
	\label{eq:euler-formula}
	y_{i+1} = y_i + h f(x_i, y_i)
\end{align}
Полученная \cref{eq:euler-formula} --- рекуррентная формула вывода звеньев ломаной Эйлера. \(x_0, y_0\) выделяются из начальных условий.

Для поставленной задачи рекуррентная формула \cref{eq:euler-formula} имеет вид
\begin{align}
	y_{i+1} = y_i + h y_i \cos x_i
\end{align}

Оценка остатка идет из разложения в ряд Тейлора:
\begin{align}
	\overline{y}_{i+1} = y_i + h y'(x_i, y_i) + \frac{h^2}{2!} y''(x_i, y_i) + o(h^3)
\end{align}
Учит

\subsubsection{Метод Рунге-Кутта}
Формула Рунге-Кутта четвертого порядка точности:
\begin{align}
	\begin{cases}
		k_{1i} = hf(x_i, y_i)                                             \\
		k_{2i} = hf\left(x_i + \frac{h}{2}, y_i + \frac{k_{1i}}{2}\right) \\
		k_{3i} = hf\left(x_i + \frac{h}{2}, y_i + \frac{k_{2i}}{2}\right) \\
		k_{4i} = hf(x_i + h, y_i + k_{3i})
	\end{cases} \\
	y_{i+1} = y_i + \frac{1}{6}(k_{1i} + 2 k_{2i} + 2 k_{3i} + k_{4i})
\end{align}

А для заданного примера коэффициенты принимают вид
\begin{align}
	\begin{cases}
		k_{1i} = h y_i \cos x_i                                                            \\
		k_{2i} = h \left(y_i + \frac{k_{1i}}{2} \right)\cos\left(x_i + \frac{h}{2} \right) \\
		k_{3i} = h \left(y_i + \frac{k_{2i}}{2} \right)\cos\left(x_i + \frac{h}{2} \right) \\
		k_{1i} = h (y_i + k_{3i}) \cos (x_i + h)                                           \\
	\end{cases}
\end{align}
Погрешность метода есть величина порядка \(h^5\) (в предположении, что \(f(x,y) \in C^{(5)}\))

Для определения правильности выбора шага \(h\) на каждом этапе применяется двойной пересчет: исходя из текущего значения \(y(x_i)\), вычисляется \(y(x_i + 2h)\) двумя способами:
\begin{itemize}
	\item С шагом \(h\), проходя две ступени;
	\item С шагом \(2h\).
\end{itemize}
Если расхождение полученных результатов не превышает заданной погрешности, то \(h\) выбран верно --- полученное значение можно принять за \(y(x_i + 2h)\). В противном случае шаг уменьшается в два раза.

В работе приводится выбор фиксированного шага и показанный динамический, основанный на двойном пересчете.

\subsubsection{Метод Адамса}
Выводится из второй интерполяционной формулы ньютона и соображения
\begin{align}
	\Delta y_i = \int_{x_i}^{x_{i+1}} y'(x) dx                                                                                                         \\
	y'(x) = y'_i + q \Delta y'_{i-1} + \frac{q(q + 1)}{2!} \Delta^2 y'_{i-2} + \frac{q(q+1)(q+2)}{3!}\Delta^3 y'_{i-3}                                 \\
	\Delta y_i = h \int_0^1\left( y'_i + q \Delta y'_{i-1} + \frac{q(q + 1)}{2!} \Delta^2 y'_{i-2} + \frac{q(q+1)(q+2)}{3!}\Delta^3 y'_{i-3}\right) dq \\
	\Delta y_i = hy'_{i-1} + \frac{1}{2}\Delta(h y'_{i-1}) + \frac{5}{12}\Delta^2(hy'_{i-2}) + \frac{3}{8}\Delta^3(hy'_{i-3})
\end{align}
И тогда
\begin{align}
	y_{i+1} = y_i + hy'_{i-1} + \frac{1}{2}\Delta(h y'_{i-1}) + \frac{5}{12}\Delta^2(hy'_{i-2}) + \frac{3}{8}\Delta^3(hy'_{i-3})
\end{align}

Из того, что
\begin{align*}
	\Delta y'_{i-1} = y'i - y'_{i-1}               \\
	\Delta^2 y'_{i-2} = y'i - 2y'_{i-1} + y'_{i-2} \\
	\Delta^3 y'_{i-3} = y'i - 3y'_{i-1} + 3y'_{i-2} - y'{i-3}
\end{align*}
Выводится
\begin{align}
	y_i{i+1} = y_i + \frac{h}{24}(55 y'_i - y'_{i-1} + 37 y'_{i - 2} - 9 y'_{i-3}) \\
\end{align}

Для начальных значений \(y'_i, i = \overline{0, 3}\) выбираются значения, полученные другим методом (Рунге-Кутта).

