\section{Цель работы}
научиться находить приближённое решение краевой задачи для дифференциальных уравнений в частных производных гиперболического типа методом сеток на ЭВМ.

\section{Содержание работы}
\begin{enumerate}
	\item Изучить метод сеток для дифференциальных уравнений в частных производных параболического типа с использованием явной и неявной схемы;
	\item Заменить исходное уравнение, используя конечно-разностную аппроксимацию по явной и неявной схеме;
	\item Составить программу для получения приближённого численного решения краевой задачи для уравнения теплопроводности однородного стержня, используя явную и неявную схемы.
	\item Решить задачу с помощью ЭВМ.
	\item Сравнить результаты, полученные с помощью явной и неявной схем при различных значениях \(S\).
	\item Составить отчёт о работе.
\end{enumerate}

\section{Выполнение работы}
\subsection{Задание}
Получить приближённое численное решение краевой задачи, используя функции
\begin{align}
	f(x) = \gamma e^{m x} + \cos{\gamma x}, \\
	\varphi(t) = \alpha t + \sin{\beta t},  \\
	\psi(t) = e^{N t} + M \sin{N + m t}
\end{align}

Провести вычисления значений функции \(u(x,t)\) с помощью составленной программы, если \(l = 1\), \(n = 10\), \(j = \overline{0, 10}\). Использовать в неявной схеме три значения \(S\colon S=6, S>6, 0<S<6\).

\subsection{Решение}
Пусть дано уравнение теплопроводности
\begin{align} \label{eq:init-u-diff}
	\frac{\partial u}{\partial t} = a^2 \frac{\partial^2 u}{\partial x^2}
\end{align}
где \(u(x,t)\) ---  температура стержня в точке с координатой \(x\) в момент времени \(t\).
С начальным условием:
\begin{align}\label{eq:init-cond}
	u(x, 0) = f(x)
\end{align}

И граничными условиями:
\begin{align}\label{eq:bnd-cond}
	\begin{cases}
		u(0,t) = \varphi(t), \\
		u(l,t) = \psi(t)
	\end{cases}
\end{align}

Требуется найти распределение температуры \(u(x,t)\) вдоль стержня в любой точке \(x\in[0,l]\) момент времени \(t\). Для решения этой задачи используется метод сеток.

\subsubsection{Явная схема метода сеток для уравнений параболического типа}

В методе сеток в полуполосе \(t \geq 0, 0 \leq x \leq l\) строится сетка, узлами которой являются точки \((x_i, t_j)\), где \(x_i = i \cdot h\), \(i = \overline{0,n}\), \(t_j=j\cdot k\), \(j = 0,1,2\dots\). Шаги \(h\) и \(k\) по осям \(x\) и \(t\) выбираются так:
\begin{align}
	h = \frac{l}{n}, k = \frac{h^2}{6}
\end{align}

Где \(l\) --- длина стержня.

Приняв \(u_{i,j} = u(x_i,t_j)\) и заменив \cref{eq:init-u-diff}, конечно-разностными соотношениями, получается
\begin{align}
	\frac{u_{i,j+1} -u_{i,j}}{h^2/6} = \frac{u_{i+1,j} - 2 u_{i,j} + u_{i - 1,j}}{h^2}, \forall i = \overline{0,n}, j = 0,1,2,\dots
\end{align}

После преобразования:
\begin{align}\label{eq:res-formula}
	u_{i,j+1} = \frac{1}{6}(u_{i-1, j} + u_{i,j} + u_{i-1, j})
\end{align}

Таким образом, для подсчёта значения искомой функции \(u(x_i, t_j)\) в узловых точках слоя \((j + 1)\) используются уже известные значения этой функции в трёх узловых точках слоя \(j\). Исходя из значений \(u(x_i, t_j)\) на начальном слое в момент времени \(t = 0\), значения функции \(u(x_i, t_j)\) для которого определяются из начального условия
\begin{align}
	u(x_i, 0)=f(x_i), i = \overline{1,n}
\end{align}

А также используя значения функции в граничных узловых точках, определяемые граничными условиями \cref{eq:bnd-cond} по формуле \cref{eq:res-formula} последовательно вычисляются значения \(u(x_i, t_j)\).

\subsubsection{Неявная схема метода сеток для решения уравнения теплопроводности}
Шаг \(h\) по оси \(x\) выбирается так же, как и в явной схеме. Соотношение шагов:
\begin{align}
	k = \frac{h^2}{S}
\end{align}

Исходное дифференциальное уравнение \cref{eq:init-u-diff} заменяется конечно-разностными уравнениями вида:
\begin{align}\label{eq:linear-system-lk}
	\frac{u_{i,j+1} - u_{i,j}}{h^2/S} = \frac{u_{i+1,j+1} - 2u_{i,j+1} + u_{i-1,j+1}}{h^2}
\end{align}

Начальное и граничные условия остаются теми же, что в предыдущем случае. Для решения системы линейных алгебраических уравнений \cref{eq:linear-system-lk} применяется метод прогонки.

\begin{enumerate}
	\item Вычисляются из начальных условий \cref{eq:init-cond} значения
	      \begin{align}
		      u_{i,0} = f_i
	      \end{align}
	\item Выбирается значение \(S\) с целью получения требуемой скорости продвижения по оси \(t\).
	\item Прямой ход --- при заданном слое \(j+1\) вычисляются
	      \begin{align}
		      a_{1,j+1} = \frac{1}{2+S}              \\
		      b_{1,j+1} = \varphi_{j+1} + S u_{1, j} \\
		      a_{i,j+1} = \frac{1}{2+S-a_{i-1,j+1}}  \\
		      b_{i,j+1} = a_{i-1,j+1}b_{i-1,j+1}+S u_{i,j}, i = \overline{2, n-1}
	      \end{align}
	\item Обратный ход (для заданного слоя \(j+1\) )
	      \begin{align}
		      u_{i,j+1}(b_{i,j+1}+u_{i+1,j+1})
	      \end{align}
\end{enumerate}

Величина \(u_{n,j+1} = \psi_{j+1}\) --- значение исходной функции в точке \((x_n, t_{j+1})\), а \(u_{0,j+1} = \varphi_{j+1}\) --- в точке \((x_0, t_{j+1})\). Схема устойчива при \(S > 0\).

\subsubsection*{Для рассматриваемого примера}

\begin{align}
	f(x) = -1.7 e^{-1.7 x} + \cos(-1.7 x), \\
	F(x) = -0.5 e^x + 0.4 \cos(-1.7 x),    \\
	\varphi(t) = -0.5 t + \sin{0.4 t},     \\
	\psi(t) = e^{-0.5 t} + 0.1 \sin(-1.7 t) - 0.5
\end{align}

Начальные  условия
\begin{align}
	u(x, 0) = -1.7 e^{-1.7 x} + \cos(-1.7 x), \\
	u_t'(x, 0)= -0.5 e^x + 0.4 \cos(-1.7 x)
\end{align}

Граничныме условия
\begin{align}
	\begin{cases}
		u(0,t) = -0.5 t + \sin{0.4 t}, \\
		u(l,t) = e^{-0.5 t} + 0.1 \sin(-1.7 t) - 0.5
	\end{cases}
\end{align}

Начальное условие в конечно-разностных соотношениях
\begin{align}
	u_{i,0} = -1.7 e^{-1.7 x_i} + \cos(-1.7 x_i) \\
	u_{i,-1} = u_{i,0} - k F_i = -1.7 e^{-1.7 x_i} + \cos(-1.7 x_i) - k (-0.5 e^x + 0.4 \cos(-1.7 x)
\end{align}

Граничные условия в конечно-разностных соотношениях:
\begin{align}
	u_{0,j} = -0.5 t_j + \sin{0.4 t_j}, \\
	u_{n,j} = e^{-0.5 t_j} + 0.1 \sin(-1.7 t_j) - 0.5
\end{align}

\subsection{Листинг программы}
\inputminted[frame=lines, linenos]{cpp}{listings/main.cc}

\subsection{Прямое вычисление интеграла}

\begin{multline}
	\iint\limits_D(2x+y)dxdy = \int_2^6\int_{8/x}^{2x} (2x+y) dy dx = \int_2^6 \roundbrac{2xy\Bigr|_{8/x}^{2x} + \frac{y^2}{2}\Bigr|_{8/x}^{2x} }dx = \\
	= \int_2^6 \roundbrac{4x^2 - 16 + \frac{4x^2}{2} - \frac{64}{2x^2}} = \int_2^6\roundbrac{
		2x^2-16-\frac{32}{x^2}}dx = 2x^3\Bigr|_2^6 - 16x\Bigr|_2^6 + \frac{32}{x}\Bigr|_2^6 =\\
	= \frac{1024}{3} \approx 341.(3)
\end{multline}


\section{Вывод}
Составлена программа, позволяющая вычислять значение функции \(u(x,t)\), которая определяет положение точки \(x_i\) струны длины \(l\) в любой момент времени \(t_j\).

