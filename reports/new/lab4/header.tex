\newpagestyle{titlePS}{
	\setfoot{}{Казань 2025}{}
}
\thispagestyle{titlePS}

\begin{center}
	\MakeUppercase{ Министерство науки и высшего образования российской федерации }\\
	федеральное государственное бюджетное образовательное учреждение высшего образования <<Казанский национальный исследовательский
	технический университет им. А.Н. Туполева-КАИ>>\\
	(КНИТУ-КАИ)
\end{center}

\(\underset{\text{(наименование института)}}{\underline{\text{Институт Компьютерных технологий и защиты информации}\hspace{2cm}}}\)

Кафедра \( \underset{\text{(наименование кафедры)}}{\underline{\text{Прикладной математики и информатики}\hspace{2cm}}} \)
\vspace{0pt plus2fill}
\begin{center}
	\textbf{\MakeUppercase{отчет}}\\
	\textbf{по лабораторной работе 3} \\
	\ulwt{1em}{Приближенное решение краевой задачи для дифференциальных уравнений}{1em}
	\ulwt{1em}{в частных производных гиперболического типа методом сеток}{1em}
\end{center}
\vspace{0pt plus1fill}

\begin{center}
	Номер индивидуального задания: 7
\end{center}

\vspace{0pt plus2fill}
\hfill\parbox{9cm}{
	Выполнил студент группы \underline{4312} \\
	\( \underset{\text{(ФИО)}}{\underline{\hspace*{0em}\text{Д.Д.Наумихин}\hspace*{1em}}} \) \vspace{1em} \\
	%Руководитель практики от университета \vspace{1em} \\
	%Проверил \vspace{1em}\\
	%\( \underset{\text{(должность)}}{\underline{\text{доцент}}} \) \hfill \( \underset{\text{(подпись)}}{\underline{\text{\phantom{,}}\hspace{2cm}}} \) \hfill \( \underset{\text{(расшифровка подписи)}}{\underline{\text{Комиссарова Е.М.}}}\) \vspace{1em} \\
	%Отчёт защищён с оценкой: \hrulefill \vspace{1ex} \\
	%Дата защиты <<\underline{\hspace{2ex}}>> \hrulefill \hspace{1ex} 2024 г.
}

\vspace{0pt plus2fill}

