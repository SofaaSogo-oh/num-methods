\section{Выводы}
\begin{enumerate}
	\item Получаемая оценка интеграла с помощью метода Монте-Карло --- случайная величина. Как случайная величина она имеет математическое ожидание, дисперсию и стандартное отклонение.
	\item Сходится данный метод крайне медленно, теоретическая оценка в \(O(1/\sqrt{n})\) совпадает с полученными результатами с найденных средних оценках нормального отклонения случайной величины от предполагаемого истинного. Увеличение числа точек ведет к уменьшению нормального отклонения случайной величины приближения величины интеграла.
	\item Сравнивается два способа вычисления двойного интеграла:
	      \begin{enumerate}
		      \item Через Выраженные случайные величины --- отображения из \([0,1]^3\) (\cref{method-1})
		      \item Через Выраженные случайные величины --- отображения из \([0,1]^2\) (\cref{method-2});
	      \end{enumerate}
	      Эти два метода имеют одну и ту же асимптотическую оценку отклонения метода, но метод, приводимый в \cref{method-2} в целом имеет меньшее отклонение.
\end{enumerate}
