\subsection{Введение вероятностной модели}\label{method-1}
При условии, что \(f\circ\varphi\) неотрицательно на области \(D'\), данное тождество истинно:
\begin{align}
	f(\varphi(t)) = \int_0^{f(\varphi(t))}dr.
\end{align}
Допустив
\begin{align}
	0 \leq f(\varphi(t)) \leq R \Longrightarrow 0 \leq \frac{f(\varphi(t))}{R} \leq 1, \forall t \in D',
\end{align}
можно сделать замену
\begin{align}
	u = r R \Longrightarrow du = R dr \\
	r_a = 0, r_b = f(\varphi(t))/R.
\end{align}
Интеграл (\ref{eq:ab-iint}) принимает вид:
\begin{align}
	I = (A - a)(B - b)R\iint\limits_{D'}\int_0^{f(\varphi(t))/R}drdt = (A - a)(B - b)R \iiint\limits_{D''}d\xi,
\end{align}
где \(D'' \subset [0,1]^3\) находится над гиперплоскостью \(z = 0\) и под гиперповерхностью \(z =f(\varphi(t))\)

Для вероятностного пространства \((\Omega, F, P)\):
\begin{enumerate}
	\item \(\Omega = [0,1]^3 \subset \mathbb{R}^3\) --- множество элементарных событий;
	\item \(F\) --- \(\sigma\) алгебра подмножеств множества \(\Omega\), для которых введена мера \(\mu\);
	\item Вероятность \(P\) задается следующим образом:
	      \begin{align*}
		      P(X) = \frac{\mu(X)}{\mu(\Omega)} = \frac{\mu(X)}{1} = \mu(X)
	      \end{align*}
\end{enumerate}
Задача оценки интеграла как меры сводится к оценке вероятности попадания случайно выбранной точки из \(\Omega\) в область \(D''\).

Согласно теореме Бернулли, частота \(p^*\) появления события \(A\) сходится по вероятности к вероятности \(p\):
\begin{align}
	\lim_{n\to\infty}P\{|p^* - p| < \varepsilon \} = 1
\end{align}

Пусть \(X_i\) --- индикаторная случайная величина \(I_{D''}(X_i)\). Тогда:
\begin{align}
	P(I_{D''}(X_i)=1) = \mu(D'') = I_0 \\
	M[X_i] = 0(1 - I_0) + 1 I_0 = I_0  \\
	D[X_i] = (0 - I_0)^2 (1 - I_0) + (1 - I_0)^2 I_0 = I_0(1-I_0) < \frac{1}{4}
\end{align}
Задав случайную величину \(S_n\):
\begin{align}
	p^* = \frac{1}{n}\sum_{i=0}^n P(I_{D''}(X_i) = 1) = S_n,
\end{align}
выходит:
\begin{align}
	M[S_n] = \frac{1}{n}M[\sum_{i=0}^n P(I_{D''}(X_i) = 1)] = \frac{1}{n}\sum_{i=0}^n M[P(I_{D''}(X_i) = 1)] = \frac{1}{n}\sum_{i=0}^n I_0 = \frac{n}{n}I_0 = I_0 \\
	D[S_n] = \frac{1}{n^2}D[\sum_{i=0}^n P(I_{D''}(X_i) = 1)] =  \frac{1}{n^2}\sum_{i=0}^n I_0(1-I_0) = \frac{(1 - I_0)I_0}{n} < \frac{1}{4n}
\end{align}
И неравенство Чебышева
\begin{align}
	P\{|X-m_x|<\varepsilon\} \geq 1 - \frac{D_x}{\varepsilon^2}
\end{align}
принимает вид:
\begin{align}
	P\{|p^* - p|<\varepsilon\} \geq 1 - \frac{I_0(1-I_0)}{\varepsilon^2 n} \geq 1 - \frac{1}{4 \varepsilon^2 n} \\
	\label{eq:garant}
	P\{|p^* - p|<\varepsilon\} = 1 - \delta
\end{align}
Приняв для данной оценки \(\varepsilon\) гарантийную вероятность (\ref{eq:garant}), выводится, что условие имеет место быть при
\begin{align}
	\frac{1}{4 \varepsilon^2 n} = \delta \Longrightarrow \varepsilon = \frac{1}{2\sqrt{\delta n}}.
\end{align}
Итак, поведение величины \(\varepsilon\) описывается
\begin{align}\label{eq:o-desc}
	\varepsilon = O\left(\frac{1}{\sqrt{n}}\right).
\end{align}
И необходимое число испытаний тогда:
\begin{align}
	n = \frac{1}{4 \varepsilon^2 \delta}
\end{align}
