\subsection{Исключение одной случайной величины}\label{method-2}
Если взять пространство \((\Omega, F, P)\) с
\begin{enumerate}
	\item \(\Omega = [0,1]^2 \subset R^2\);
	\item \(F\) --- \(\sigma\) алгебра подмножеств множества \(\Omega\), для которых введена мера \(\mu\);
	\item Вероятность \(P\) задается следующим образом:
	      \begin{align*}
		      P(X) = \frac{\mu(X)}{\mu(\Omega)} = \frac{\mu(X)}{1} = \mu(X)
	      \end{align*}
\end{enumerate}

и рассмотреть случайную велчину вида
\begin{align}
	Y = I_{D'}(X)f(X),
\end{align}
то получается:
\begin{align}
	M[Y] = \iint_{[0,1]^2} p(X) I(X) f(X) dX.
\end{align}

Из определения индикатора:
\begin{align}
	M[Y] = \iint_{D'} p(X)f(X) dX,
\end{align}
учитывая равномерное распределение
\begin{align}
	p(x) = \frac{1}{A} = 1,
\end{align}
выходит
\begin{align}
	M[Y] = \iint_{D'} f(X) dX = I_0.
\end{align}
Дисперсия вычисляется как
\begin{align}
	D[Y] = \mu_2[Y] = \alpha_2[Y] - (\alpha_1[Y])^2 = M[Y^2] - (M[Y])^2 \\
	M[Y^2] = \iint_{[0, 1]^2} p(X)(I_{D'}(X)f(X))^2 dX = \iint_{D'}f^2(X) dX
\end{align}

Случайная величина, заданная как сумма данных независимых случайных величин
\begin{align}
	S = \frac{1}{n}\sum_{k=0}^n Y
\end{align}
имеет характеристики:
\begin{align}
	M[S] = \frac{n}{n}I_0 = I_0 \\
	D[S] = \frac{n}{n^2} D[Y] = \frac{1}{n} D[y]
\end{align}

Подставляя в неравеснтво Чебышева:
\begin{align}
	P\{|S - I_0|<\varepsilon\} \geq 1 - \frac{\iint_{D'}f^2(X)dX - I_0^2}{\varepsilon^2 n}. \\
\end{align}

Как итог
\begin{align}
	\lim_{n\to\infty}P\{|S - I_0|<\varepsilon\} = 1.
\end{align}
Асимптотически величина \(\varepsilon\) описывается так же, как и в \cref{eq:o-desc}.
