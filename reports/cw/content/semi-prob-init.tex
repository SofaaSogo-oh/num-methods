\section{Исключение одной случайной величины}
Если взять уже пространство с
\begin{enumerate}
	\item \(\Omega = [0,1]^2 \subset R^2\);
	\item \(F\) то же;
	\item \(P\) то же
\end{enumerate}

И рассмотреть случайную велчину вида
\begin{align}
	Y = I_{D'}(X)f(X)
\end{align}

То получается:
\begin{align}
	M[Y] = \iint_{[0,1]^2} p(X) I(X) f(X) dX
\end{align}
При том выходит, из определения индикатора:
\begin{align}
	M[Y] = \iint_{D'} p(X)f(X) dX
\end{align}
А учитывая равномерное распределение, выходит, что
\begin{align}
	p(x) = \frac{1}{A} = 1
\end{align}
И тогда
\begin{align}
	M[Y] = \iint_{D'} f(X) dX = I_0
\end{align}
Дисперсия вычисляется как
\begin{align}
	D[Y] = \mu_2[Y] = \alpha_2[Y] - (\alpha_1[Y])^2 = M[Y^2] - (M[Y])^2 \\
	M[Y^2] = \iint_{[0, 1]^2} p(X)(I_{D'}(X)f(X))^2 dX = \iint_{D'}f^2(X) dX
\end{align}
И это выходит конечное число.

И если взять другую случайную величину как сумму данных независимых величин:
\begin{align}
	S = \frac{1}{n}\sum_{k=0}^n Y
\end{align}
Вычисляя основные величины:
\begin{align}
	M[S] = \frac{n}{n}I_0 = I_0 \\
	D[S] = \frac{n}{n^2} D[Y] = \frac{1}{n} D[y]
\end{align}

И в итоге записывая неравенство Чебышева:
\begin{align}
	P\{|S - I_0|<\varepsilon\} \geq 1 - \frac{\iint_{D'}f^2(X)dX - I_0^2}{\varepsilon^2 n} \\
\end{align}

В итоге выходит, что
\begin{align}
	\lim_{n\to\infty}P\{|S - I_0|<\varepsilon\} = 1
\end{align}
И асимптотически величина \(\varepsilon\) описывается так же, как и в \cref{eq:o-desc}.
