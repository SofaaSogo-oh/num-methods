\section{Результаты работы программы}
\subsection{Пример вычисления}
Для \(N = 20\) рассматривается выборка точек с ее содержимым:

\begin{align*}
	\begin{array}{ll}
		(0.56763554,0.0789485) \to 457.23883   & (0.74630606,0.6721467) \to 788.1961   \\
		(0.43813038,0.57967687) \not\in D      & (0.34509814,0.5507567) \not\in D      \\
		(0.14233947,0.4783113) \not\in D       & (0.032078028,0.31770563) \not\in D    \\
		(0.8451768,0.15828753) \to 588.081     & (0.37217999,0.48739147) \to 576.4103  \\
		(0.48926282,0.085366964) \to 433.40872 & (0.2324897,0.54943454) \not\in D      \\
		(0.8560116,0.7436471) \to 858.1829     & (0.4741887,0.977206) \not\in D        \\
		(0.6843517,0.65986276) \to 761.45844   & (0.79020524,0.92588997) \not\in D     \\
		(0.17501795,0.9010966) \not\in D       & (0.59115267,0.17000258) \to 506.70572 \\
		(0.51510084,0.84061956) \not\in D      & (0.8522557,0.82952595) \to 895.9853   \\
		(0.8507035,0.46353948) \to 728.8909    & (0.56606627,0.77454865) \not\in D     \\
	\end{array}
\end{align*}
\begin{align*}
	I = \frac{1}{20}\sum_{k=0}^{10}=\frac{1}{20}( & 457.23883+588.081+433.40872+                     \\
	+                                             & 858.1829+761.45844+728.8909+                     \\
	+                                             & 788.1961+576.4103+506.70572+895.9853) = 329.7279
\end{align*}
Так как представляемое значение интеграла является случайной величиной будет неплохо найти приближения характеристик этой случайной величины. А именно математическое ожидание, дисперсия и стандартное отклонение.
\begin{enumerate}
	\item Генерируется с указанным \(N\) выборки приближенного вычисления интеграла размером \(m=100\);
	\item Находится выборочное среднее
	      \begin{align}
		      \overline{m}_x = \frac{1}{m}\sum_{i=1}^m x_i
	      \end{align}
	\item Находится исправленная выборочная дисперсия
	      \begin{align}
		      S^2 = \frac{1}{m-1}\sum_{i=1}^m (x_i - \overline{m}_x)^2
	      \end{align}
	\item Находится оценка стандартного отклонения:
	      \begin{align}
		      S = \sqrt{S^2} = \sqrt{\frac{1}{m-1}\sum_{i=1}^m(x_i-\overline{m}_x)^2}
	      \end{align}
\end{enumerate}

Приводится сводная таблица по вычисленным приближениям:
\begin{enumerate}
	\item При вычислении по \cref{method-1}
	      \begin{table}[h!]
		      \centering
		      \begin{tabular}{|l|cccc|}
			      \hline
			      $N$   & $\overline{m}_x$ & $S^2$       & $S$         & $|\overline{m}_x - I|$ \\
			      \hline
			      $20$  & $342.528$        & $10679.484$ & $103.34159$ & $1.1946666$            \\
			      $50$  & $334.6432$       & $5202.782$  & $72.13032$  & $6.6901336$            \\
			      $100$ & $340.48$         & $2009.858$  & $44.83144$  & $0.85333335$           \\
			      $200$ & $344.576$        & $870.0559$  & $29.49671$  & $3.2426667$            \\
			      $300$ & $342.18668$      & $841.04535$ & $29.000782$ & $0.85333335$           \\
			      \hline
		      \end{tabular}

	      \end{table}

	\item При вычислении по \cref{method-2} выходит:
	      \begin{table}[h!]
		      \centering
		      \begin{tabular}{|l|cccc|}
			      \hline
			      $N$   & $\overline{m}_x$ & $S^2$       & $S$         & $|\overline{m}_x - I|$ \\
			      \hline
			      $20$  & $340.4964$       & $5191.2344$ & $72.050224$ & $0.8369446$            \\
			      $50$  & $345.91507$      & $2315.847$  & $48.12325$  & $4.581726$             \\
			      $100$ & $346.79022$      & $1288.8688$ & $35.900818$ & $5.4568787$            \\
			      $200$ & $344.6939$       & $593.94995$ & $24.371088$ & $3.3605652$            \\
			      $300$ & $342.24014$      & $344.49103$ & $18.560469$ & $0.9067993$            \\
			      \hline
		      \end{tabular}
	      \end{table}
\end{enumerate}

\subsection{Отклонения}
Взяв для каждой конфигурации, для каждого \(N\), выборки в количестве \(K\) с размером каждой выборки \(K_i = K_0\) производятся вычисления:
\begin{enumerate}
	\item Вычисляется нормальное отклонение для каждой выборки
	      \begin{align}
		      \sigma_i = \sqrt{\frac{1}{K_i-1}\sum_{j=0}^{K_i} (x_k - \overline{x})^2}, i=\overline{0,K}
	      \end{align}
	\item Находится среднее значение среди полученных нормальных отклонений:
	      \begin{align}
		      m_\sigma = \frac{1}{K}\sum_{i=0}^K \sigma_i
	      \end{align}
	\item А так же нормальное отклонение среди полученных:
	      \begin{align}
		      \sigma_\sigma = \sqrt{\frac{1}{K-1}\sum_{i=0}^K(\sigma - \sigma_i)^2}
	      \end{align}
\end{enumerate}
Для вычислений взяты значения \(K=30\), \(K_0=100\).
\begin{table}[h!]
	\centering
	\caption{Подсчет оценок стандартных отклонений для методов \cref{method-1} и \cref{method-2}}
	\label{table:sigma-values}
	\begin{tabular}{|l|cccc|}
		\hline
		$N$     & $m_{\sigma1}$ & $\sigma_{\sigma1}$ & $m_{\sigma2}$  & $\sigma_{\sigma2}$ \\
		\hline
		$  20 $ & $  74.08163 $ & $ 4.6014333 $      & $ 109.821976 $ & $  7.773065 $      \\
		$  50 $ & $  46.76052 $ & $ 3.2611177 $      & $    68.9947 $ & $ 4.8407636 $      \\
		$ 100 $ & $ 32.825172 $ & $   2.68074 $      & $  47.794693 $ & $ 3.8238974 $      \\
		$ 200 $ & $ 22.975714 $ & $  1.596667 $      & $  34.144722 $ & $ 3.1301556 $      \\
		$ 300 $ & $ 19.188334 $ & $ 1.2513237 $      & $  27.316946 $ & $ 2.0922694 $      \\
		$ 400 $ & $ 17.110592 $ & $ 1.4182564 $      & $  24.498312 $ & $ 1.6277343 $      \\
		$ 500 $ & $ 14.788441 $ & $ 0.9899412 $      & $  21.324444 $ & $ 1.7795098 $      \\
		\hline
	\end{tabular}
\end{table}
\begin{figure}[h!]
	\hspace{-2em}
	\includegraphics[width=\linewidth]{graphics/desmos-graph.png}
	\caption{Размещены значения, указанные в \cref{table:sigma-values}}
	\label{fig:sigma-graph}
\end{figure}

На \cref{fig:sigma-graph} :
\begin{itemize}
	\item Черными точками обозначаются значения \(m_{\sigma2}\), полученные по \cref{method-1}.
	\item Белыми кружками с черным контуром обозначаются значения \(m_{\sigma1}\), полученные по \cref{method-2}.
\end{itemize}

% +-----+-----------+-----------+
% | N   | mₓ        | dₓ        |
% +-----+-----------+-----------+
% |  20 |  74.08163 | 4.6014333 |
% |  50 |  46.76052 | 3.2611177 |
% | 100 | 32.825172 |   2.68074 |
% | 200 | 22.975714 |  1.596667 |
% | 300 | 19.188334 | 1.2513237 |
% | 400 | 17.110592 | 1.4182564 |
% | 500 | 14.788441 | 0.9899412 |
% +-----+-----------+-----------+
% +-----+------------+-----------+
% | N   | mₓ         | dₓ        |
% +-----+------------+-----------+
% |  20 | 109.821976 |  7.773065 |
% |  50 |    68.9947 | 4.8407636 |
% | 100 |  47.794693 | 3.8238974 |
% | 200 |  34.144722 | 3.1301556 |
% | 300 |  27.316946 | 2.0922694 |
% | 400 |  24.498312 | 1.6277343 |
% | 500 |  21.324444 | 1.7795098 |
% +-----+------------+-----------+
