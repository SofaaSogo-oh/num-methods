\section{Решение}
\subsection{\RomanNum{1} интерполяционная формула Ньютона}
\subsubsection{Теоретический вывод}
Задача такова: необходимо с помощью набора простых функций (можно назвать базисом), состоящим из функций
\begin{align}
	\{e_i\}	= \{1\} \cup \left\{\prod_{i=0}^k (x - x_i)\colon k=\overline{0,n}\right\}
\end{align}
Из равноудаленности точек, данные базисные функции можно записать кратко:
\begin{align}
	x^{[k]} = & \begin{cases}
		            \prod_{i=0}^{k - 1} (x - i h), & k > 0 \\
		            1,                             & k = 0
	            \end{cases},\, k = \overline{0,n}                    \\
	\{e_i\} = & \left\{ (x - x_0)^{[k]}\colon k = \overline{0,n} \right\}
\end{align}

Далее мы должны найти такую их линейную комбинацию, что в заданных равностоящих узлах. Этот полином принимает целевые значения:
\begin{align}\label{eq:pf_eq}
	P_n(x_i) = f(x_i) = y_i
\end{align}
Шаблон полинома:
\begin{align}
	P(x) = \sum_{k=0}^n a_k (x - x_0)^{[k]}
\end{align}
При последовательной подстановке точек $x_i$ в определение \cref{eq:pf_eq} можно найти искомые коэффициенты $a_k$:
\begin{align}
	 & x = x_i \Longrightarrow x - x_i = 0                                                                            \\
	 & \forall i,j = \overline{0,n}\colon\, j > i \Longrightarrow \prod_{k=0}^{j - 1} (x - x_k) = (x - x_0)^{[j]} = 0
\end{align}
Аналогично:
\begin{align}
	x            = x_i                   \\
	\forall i,j  = \overline{0,n}\colon\, j \leq i \Longrightarrow \prod_{k=0}^{j-1} (x - x_k)  = (x_i - x_0)^{[j]} =
	(x_0 + i h - x_0)^{[j]} = (ih)^{[j]} \\
	(ih)^{[j]} = \prod_{k=0}^{j-1} (ih - kh) = \prod_{k=0}^{j-1} h (i - k) = h^j \frac{i!}{(i-j)!} = h^j \lambda_{ij}
\end{align}

Итак, формируются коэффициенты вида:
\begin{align}
	b_{ij} = \begin{cases}
		         h^j \lambda_{ij}, & j \leq i \\
		         0,                & j > i
	         \end{cases} \\
	\lambda_{ij} = \frac{i!}{(i-j)!} \label{eq:lambda_def}
\end{align}

В итоге получаем матричное уравнение:
\begin{align}
	\begin{pmatrix}
		\frac{0!}{(0-0)!} & 0                 & \cdots & 0                 \\
		\frac{1!}{(1-0)!} & \frac{1!}{(1-1)!} & \cdots & 0                 \\
		\vdots            & \vdots            & \ddots & \vdots            \\
		\frac{n!}{(n-0)!} & \frac{n!}{(n-1)!} & \cdots & \frac{n!}{(n-n)!} \\
	\end{pmatrix}
	\diag\{ h^k \colon k = \overline{0, n}\}
	\begin{pmatrix}
		a_0    \\
		a_1    \\
		\vdots \\
		a_n    \\
	\end{pmatrix} =
	\begin{pmatrix}
		y_0    \\
		y_1    \\
		\vdots \\
		y_n    \\
	\end{pmatrix}
\end{align}

Матрица получилась треугольная. Необходимо найти решение для \(a_k\). Для нахождения можно вычитать одну строку из другой, умножать на определенное число. Вычитание одной строки из другой (соседней) означает, что мы находим:
\begin{align}
	\Delta P(x_k) = P(x_k + h) - P(x_k) = P(x_{k + 1}) - P(x_k)
\end{align}
При том вычитаются и решения:
\begin{align}
	\Delta f(x_k) = f(x_k + h) - f(x_k) = f(x_{k + 1}) - f(x_k) = y_{k + 1} - y_{k}
\end{align}

Коэффициенты здесь таковы, что:
\begin{align}
	\begin{pmatrix}
		1 & k     & k(k-1) & \cdots & k!     & 0      & \cdots & 0 \\
		1 & (k+1) & (k+1)k & \cdots & (k+1)! & (k+1)! & \cdots & 0 \\
	\end{pmatrix}
\end{align}
Вычитая строки:
\begin{align}
	\begin{pmatrix}
		0 & 1 & 2k & \cdots & k!k & k! & \cdots & 0
	\end{pmatrix}
\end{align}
В общем случае:
\begin{multline}\label{eq:delta_base}
	\lambda_{(k+1)t} - \lambda_{kt} = \frac{(k+1)!}{(k+1-t)!} - \frac{k!}{(k-t)!} = \frac{(k+1)!}{(k+1-t)!} - \frac{((k+1)-t)}{((k+1)-t)}\frac{k!}{(k-t)!} = \\
	= \frac{(k+1)k!}{(k+1-t)!} - \frac{(k+1-t)k!}{(k+1-t)!} = \frac{k!(k+1-k-1+t)}{(k+1-t)!} = t \frac{k!}{(k + 1 - t)!} = \\
	= t \frac{k!}{(k + 1 - t)!} = t \frac{k!}{(k-(t-1))!} = t \lambda_{k(t-1)}
\end{multline}
То есть, мы имеем преобразование следуюещго плана:
\begin{align}
	\delta\colon \lambda_{kt} \mapsto
	\begin{cases}
		t \lambda_{k(t-1)}, & t > 0 \\
		0,                  & t=0
	\end{cases}
\end{align}
Повторное применение преобразования:
\begin{align}
	\delta^2\lambda_{kt} = \delta \delta \lambda_{kt} = \delta t \lambda_{k(t-1)} = t \delta \lambda_{k(t-1)} = t(t-1)\lambda_{k(t-2)}
\end{align}
Следующее утверждение --- результат рассуждений:
\begin{align}
	\delta^m \lambda_{kt} = \frac{t!}{(t-m)!} \lambda_{k(t-m)}
\end{align}
\begin{proof}
	Доказывается по индукции:
	\begin{enumerate}
		\item База доказана в \cref{eq:delta_base}.
		\item Переход:
		      \begin{multline}
			      \delta^{m+1}\lambda_kt = \delta \delta^{m} \lambda_kt = \frac{t!}{(t-m)!}\delta \lambda_{k(t-m)} = \\
			      = \frac{t!}{(t-m)!} (t-m) \lambda_{k(t-(m+1))} = \frac{t!}{(t-(m+1))!}\lambda_{k(t-(m+1))}
		      \end{multline}
	\end{enumerate}
\end{proof}
Из определения коэффициентов \cref{eq:lambda_def} видно, что:
\begin{align}
	\lambda_{k0} = \frac{k!}{k!} = 1
\end{align}
Значит:
\begin{align}
	\begin{pmatrix}
		1 & k & k(k-1) & \cdots & k! & 0 & \cdots & 0
	\end{pmatrix} \xmapsto{\delta^k}
	\begin{pmatrix}
		0 & 0 & 0 & \cdots & k! & \cdots & 0
	\end{pmatrix}
\end{align}
Далее:
\begin{align}
	\begin{pmatrix}
		1      & 0      & 0      & \cdots & 0      \\
		0      & h      & 0      & \cdots & 0      \\
		0      & 0      & 2 h^2  & \cdots & 0      \\
		\vdots & \vdots & \vdots & \ddots & \vdots \\
		0      & 0      & 0      & \cdots & n! h^n \\
	\end{pmatrix}
	\begin{pmatrix}
		a_0    \\
		a_1    \\
		a_2    \\
		\vdots \\
		a_n    \\
	\end{pmatrix} =
	\begin{pmatrix}
		\Delta^0 y_0 \\
		\Delta^1 y_0 \\
		\Delta^2 y_0 \\
		\vdots       \\
		\Delta^n y_0 \\
	\end{pmatrix} \\
	\begin{pmatrix}
		a_0    \\
		a_1    \\
		a_2    \\
		\vdots \\
		a_n    \\
	\end{pmatrix} =
	\begin{pmatrix}
		y_0                        \\
		\frac{\Delta y_0}{h}       \\
		\frac{\Delta^2 y_0}{2h^2}  \\
		\vdots                     \\
		\frac{\Delta^n y_0}{n!h^n} \\
	\end{pmatrix}
\end{align}
Коэффициенты найдены. Формула принимает следующий вид:
\begin{align}
	P_n(x) = \sum_{k=0}^n \frac{\Delta^k y_0}{k!h^k} (x-x_0)^{[k]}
\end{align}
Возможна замена:
\begin{align}
	q = \frac{x-x_0}{h} \Longrightarrow \frac{x-x_k}{h} = \frac{x-x_0-kh}{h} = \frac{x-x_0}{h} - k = q - k \\
	P_n(x) = \sum_{k=0}^n \frac{\Delta^k y_0}{k!}\prod_{m=0}^{k-1}(q-m)
\end{align}
\subsubsection{Формула для заданного примера}
\begin{multline}
	P = 17.999711660684547  +   2.777662935606884\, q 10^{-4} + -3.7703941724842593\, q(q - 1) 10^{-4}+ \\
	+ -2.791613731289999\,  q(q - 1)(q - 2) 10^{-7}+ 7.938734114532053 q(q - 1)(q - 2)(q - 3) 10^{-9} + \\
	+ 1.9408474827287137\,  q(q - 1)(q - 2)(q - 3)(q - 4) 10^{-11}+ \\
	+ -1.9539925233402755\,  q(q - 1)(q - 2)(q - 3)(q - 4)(q - 5) 10^{-13}
\end{multline}

