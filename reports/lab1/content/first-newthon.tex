\section{Решение}
\subsection{\RomanNum{1} интерполяционная формула Ньютона}
\subsubsection{Теоретический вывод}
Задача такова: необходимо с помощью набора простых функций (можно назвать базисом), состоящим из функций
\begin{align}
	\{e_i\}	= \{1\} \cup \left\{\prod_{i=0}^k (x - x_i)\colon k=\overline{0,n}\right\}
\end{align}
Из равноудаленности точек, данные базисные функции можно записать кратко:
\begin{align}
	x^{[k]} = & \begin{cases}
		            \prod_{i=0}^{k - 1} (x - i h), & k > 0 \\
		            1,                             & k = 0
	            \end{cases},\, k = \overline{0,n}                    \\
	\{e_i\} = & \left\{ (x - x_0)^{[k]}\colon k = \overline{0,n} \right\}
\end{align}

Далее мы должны найти такую их линейную комбинацию, что в заданных равностоящих узлах. Этот полином принимает целевые значения:
\begin{align}\label{eq:pf_eq}
	P_n(x_i) = f(x_i) = y_i
\end{align}
Собственно, сам полином имеет вид:
\begin{align}
	P(x) = \sum_{k=0}^n a_k (x - x_0)^{[k]}
\end{align}
При последовательной подстановке точек $x_i$ в определение \cref{eq:pf_eq} можно найти искомые коэффициенты $a_k$:
\begin{align}
	 & x = x_i \Longrightarrow x - x_i = 0                                                                           \\
	 & \forall i,j = \overline{0,n}\colon\, j \geq i \Longrightarrow \prod_{k=0}^j (x - x_k) = (x - x_0)^{[j+1]} = 0
\end{align}
Базис конечный --- перебрать все возможно.

С другой стороны мы получаем:
\begin{align}
	x            = x_i                   \\
	\forall i,j  = \overline{0,n}\colon\, j \leq i \Longrightarrow \prod_{k=0}^{j-1} (x - x_k)  = (x_i - x_0)^{[j]} =
	(x_0 + i h - x_0)^{[j]} = (ih)^{[j]} \\
	(ih)^{[j]} = \prod_{k=0}^{j-1} (ih - kh) = \prod_{k=0}^{j-1} h (i - k) = h^j \frac{i!}{(i-j)!}
\end{align}

\subsubsection{Формула для заданного примера}


