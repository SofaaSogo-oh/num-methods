\section{Цель работы}
Построение интерполяционных многочленов, многочленов для численного дифференцирования по заданной системе точек, используя ЭВМ.

\section{Содержание работы}
\begin{enumerate}
	\item Построение интерполяционной формулы Лагранжа, первой и второй интерполяционной формул Ньютона и апроксимационнного полинома;
	\item На заданном примере построение полиномы на ЭВМ;
	\item Составление программы на любом языке программирования (выбран ЯП семейства Lisp, диалекта Common Lisp), реализующую процесс построения указанных полиномов второго порядка для системы из \underline{равноотстоящих} узловых точек;
	\item Анализ точности построения полиномов;
	\item Составление отчета о проделанной работе.
\end{enumerate}

\section{Задание}
\subsection{Ход работы}
\begin{enumerate}
	\item Составить таблицу значений экспериментальной функции \( y = 18\sin(\sqrt{x^3} + 8) \) для равноотстоящей системы из шести узловых точек \(x_{i+1} = x_i + h, i = \overline{0,\,5}\) на отрезке \(\) из области допустимых значений функции, где h = 5;
	\item По сформированной системе точек построить интерполяционные формулы Лагранжа, \RomanNum{1} и \RomanNum{2} интерполяционные формулы Ньютона;
	\item Составить программу ена любом языке программирования (выбран язык Common Lisp), реализующую процесс построения указанных полиномов для заданной системы точек.
\end{enumerate}

